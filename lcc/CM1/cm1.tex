\documentclass{article}
\usepackage{amsfonts}
\begin{document}
\author{Pablo Saavedra}
\title{Matemática Discreta}
\section{Introducción}
\subsection{}

Muestre que en todo grupo de dos o más personas, existen siempre dos
personas que tienen exactamente el mismo número de conocidos en el grupo.
\newline\newline
\underline{\textit{Proof}}
\newline

Primero que nada dejemos en claro que nadie se conoce a sí mismo y ninguna persona conoce a otra más de una vez.

En definitiva quiero ver que en un grafo simple de dos o más vértices, dos de ellos tienen el mismo grado.
\newline\newline
Sea $G=(V,\ E)$ un grafo simple con $|V| = n \geq 2$ y $|E| = m$.
\newline\newline
Sea $d$ la función que asigna a cada vértice $v \in V(G)$ su número de vecinos:
\newline

\hspace{1cm}$d: V(G) \rightarrow \mathbb{N}_{0}$

\hspace{1cm}$v \mapsto |\{ (u, v) \in E(G) \}|$
\newline\newline
Si alguien no conoce a nadie es claro que nadie puede conocer a todos.
\textit{i.e.}, si $\exists\ v \in V\ tq\ d(v) = 0 \Rightarrow \not \exists\ u \in V\ tq\ d(u) = n - 1$
\newline
Luego, tenemos que $Im(d) \not \subseteq \{0, .., n - 1\}$ y de aquí $|Im(d)| \leq n - 1$
\newline
Como $|Dom(d)| = |V| = n$ tenemos que $d$ no puede ser biyectiva por el principio del palomar.
\newline
Por lo tanto existen $u, v \in V$ distintos tal que $d(u) = d(v)$.
\newline
Por lo tanto en un grupo de dos o más personas existen siempre dos personas que tienen
exactamente el mismo número de conocidos.

\hspace{9cm}$Q.E.D$

\end{document}
