\documentclass{article}
\usepackage[utf8]{inputenc}
\usepackage{amsmath,amssymb,amsfonts}
\usepackage[spanish]{babel}
\begin{document}
\author{Pablo Saavedra}
\title{Mismos Conocidos}
\maketitle
\pagenumbering{gobble}

\noindent
Demostración de que en todo grupo de dos o más personas existen siempre dos
personas que tienen exactamente el mismo número de conocidos en el grupo.

\bigskip

\noindent
Primero que nada asumamos que:
\begin{itemize}
	\item nadie del grupo se conoce a sí mismo;
	\item nadie del grupo conoce a una persona más de una vez;
	\item nadie del grupo conoce a una persona sin que esa persona lo conozca.
\end{itemize}

\noindent
Podemos representar entonces el grupo de personas como $G = (V, E)$, un grafo simple, donde:
\begin{itemize}
	\item $V = \{\text{personas en el grupo}\}$ y $|V| = n \geq 2$.
	\item $E = \{(u, v) : u \text{ conoce a } v ;\ u,v \in V\}$ y $|E| = m$.
\end{itemize}

\noindent
Por lo tanto y en definitiva, quiero ver que en un grafo simple de dos o más vértices, dos de ellos tienen el mismo grado.

\bigskip

\noindent
Sea $d$ la función que asigna a cada vértice de $V$ su número de vecinos:

\bigskip

$$d: V(G) \rightarrow \mathbb{N}_{0}$$

$$v \mapsto |\{ (u, v) \in E(G) \}|$$

\bigskip

\noindent
Si alguien no conoce a nadie es claro que nadie puede conocer a todos.
\noindent
Esto es, si $\exists\ v \in V\ :\ d(v) = 0 \Rightarrow \nexists\ u \in V\ :\ d(u) = n - 1$.

\bigskip

\noindent
Luego, tenemos que $Im(d) \not \subseteq \{0, .., n - 1\}$ y de aquí $|Im(d)| \leq n - 1$

\bigskip

\noindent
Como $|Dom(d)| = |V| = n$ tenemos que $d$ no puede ser biyectiva por el principio del palomar.

\bigskip

\noindent
Luego existen $u, v \in V$ distintos tal que $d(u) = d(v)$.

\bigskip

\noindent
Por lo tanto en un grupo de dos o más personas existen siempre dos personas que tienen exactamente el mismo número de conocidos.

\hspace{9cm}$\blacksquare$

\end{document}
